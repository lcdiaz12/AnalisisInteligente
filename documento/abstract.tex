En este trabajo se abordó la problemática del riesgo de volcamiento al elevar vehículos de grandes dimensiones con distribuciones de peso no uniformes. Por sus características de robustez y versatilidad Consensus fue escogido como la metodología para abordar este problema. El enfoque de una red de subsistemas que convergen a un valor de referencia en un proceso de optimización  con unas trayectorias interpretadas como un método del gradiente, es usado para analizar las trayectorias y para agregar las restricciones del sistema. 
Un modelo de actuadores hidráulicos es implementado y es conectado al sistema de consensus para analizar la interacción entre ellos ante cambios de las constantes de tiempo de cada sistema.
Los mínimos criterios de convergencia respecto al intercambio de información entre agentes son probados, simulando el sistema para la red de agentes con dichas conexiones permitiendo analizar cualitativamente el comportamiento de las trayectorias.