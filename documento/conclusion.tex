
El consensus cómo modelamiento de las dinámicas de interacción entre agentes hace que cada agente llegue a una valor común para las variables de interés. Es valioso notar que debido a qu sú dinámica está definida con base en la información de los demás, esto lo hace adaptativo ante perturbaciones o cambios en la referencia.
No es necesario que cada agente conozca y comparta información con todos los agentes, basta con cumplir que el grafo sea un árbol de alcance directo (\textit{directed spanning tree}) para que el sistema converja.
Es posible acolpar el modelo real del sistema y que en conjunto consensus-planta interacciones, tengan la trayectoria deseada y converjan a el valor de referencia definido.
El enfoque desde el punto de vista de optimización es muy valioso, ya que permite interpretar las trayectorias de los agentes como el proceso de convergencia en un método del gradiente descendente. Adicionalmente permite incluir las restricciones como las ya desarrolladas en optimización, una de las cuales se usó en este trabajo (método de la barrera).
Estas perspectivas hacen interpretables cualitativamente las dinámicas del sistema y permite diseñar las tendencias de las trayectorias.